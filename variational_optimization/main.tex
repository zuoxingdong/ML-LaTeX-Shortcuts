\documentclass{article}
\usepackage[utf8]{inputenc}

%%%%%%%%%%%%%%%%%%%%%%%%%
% Version: Feb 28, 2018 %
%%%%%%%%%%%%%%%%%%%%%%%%%


% Basic setting
\usepackage[utf8]{inputenc} % allow utf-8 input
\usepackage[T1]{fontenc}    % use 8-bit T1 fonts
\usepackage{hyperref}       % hyperlinks
\usepackage{url}            % simple URL typesetting
\usepackage{nicefrac}       % compact symbols for 1/2, etc.
\usepackage{microtype}      % microtypography
\usepackage{authblk}        % Authors and affiliations


% Bibliography management (get item from Google Scholar)
% Use: 
%   1. Create a file called 'references.bib'
%   2. Copy items from Google scholar
%   3. \cite{tamar2016value}
%   4. \printbibliography
\usepackage[
style=numeric,      % [numeric, alphabetic]
sorting=nty         % nty: name, title, year
]{biblatex}
\addbibresource{references.bib}

% Figure, tables
\usepackage{graphicx}
\graphicspath{ {img/} }     % search path of images
\usepackage{subcaption}     % Multiple figures \subfigure
\usepackage{booktabs}       % professional-quality tables


% Mathematical
\usepackage{amsmath, amssymb, amsthm, amsfonts}
\usepackage{bbm}                    % 'mathbb' for digits
\usepackage[ruled]{algorithm2e}     % Algorithms package


% Define useful commands
\newtheorem{definition}{Definition}[section]  % [section]: recount after each section
\newtheorem{theorem}{Theorem}[section]
\newtheorem{lemma}[theorem]{Lemma}            % [theorem]: same counter as theorem
\newtheorem{corollary}[theorem]{Corollary}
\newtheorem{proposition}[theorem]{Proposition}

\newtheorem{assumption}{Assumption}
\newtheorem{question}{Question}
\newtheorem{remark}{Remark}
\newtheorem{case}{Case}

\DeclareMathOperator*{\argmin}{\arg\!\min}
\DeclareMathOperator*{\argmax}{\arg\!\max}


% Math bold font
\newcommand{\x}{\mathbf{x}}
\newcommand{\X}{\mathbf{X}}


% Blackboard bold: e.g. real/complex set
\newcommand{\bN}{\mathbb{N}}  % Natural set
\newcommand{\bR}{\mathbb{R}}  % real set
\newcommand{\bC}{\mathbb{C}}  % complex set
\newcommand{\bE}{\mathbb{E}}  % expectation operator


% Calligraphic font: e.g. relations, power set, topology space
\newcommand{\cA}{\mathcal{A}}
\newcommand{\cB}{\mathcal{B}}
\newcommand{\cS}{\mathcal{S}}
\newcommand{\cF}{\mathcal{F}}
\newcommand{\cU}{\mathcal{U}}


% Commonly used math notations
\newcommand{\Var}{\mathrm{Var}}  % Variance
\newcommand{\Cov}{\mathrm{Cov}}  % Covariance
\newcommand{\T}{\top}            % better for transpose operation
%% Note: add \left and \right to automatically adjust the size, e.g. fraction arguments
% norm
\newcommand{\norm}[1]{\left\lVert {#1} \right\rVert}
% absolute value
\newcommand{\abs}[1]{\left\lvert {#1} \right\rvert}
% inner product
\newcommand{\inner}[2]{\left\langle {#1}, {#2} \right\rangle}
% expectation
\newcommand{\Exp}[2]{\mathbb{E}_{{#2}} \left[  {#1} \right]}
% conditional expectation
\newcommand{\CondExp}[3]{\mathbb{E}_{{#3}} \left[  {#1} \middle| {#2} \right]}
% KL divergence
\newcommand{\KL}[2]{D_{\mathrm{KL}} \left( {#1} \| {#2} \right)}
% max KL divergence
\newcommand{\KLmax}[2]{D_{\mathrm{KL}}^{\max} \left( {#1} \| {#2} \right)}


%%%%%%%%%%%%%%%%%%%%%%%%%%%%%%%%%%%%%%%
% Paper-specific shortcuts as follows %
%%%%%%%%%%%%%%%%%%%%%%%%%%%%%%%%%%%%%%%


\title{Variational Optimization \cite{staines2012variational}}
\author{Xingdong Zuo}
\affil[]{IDSIA, Switzerland}
\date{\today}

\begin{document}

\maketitle

\section{Introduction}
Variational optimization (VO) is a technique to optimize a non-differentiable or discrete objective function with a differentiable bound on the optima. Let $x\in\ccD\subset\bbR^n$ be a vector in the domain and let $f(x)$ be a non-differentiable function, our goal is to maximize the function, i.e. $f^* = \max_{x\in\ccD} f(x)$. We cannot use traditional gradient ascent techniques due to non-differentiability of $f$. In VO, we could replace it with a differentiable surrogate objective by a lower bound on the optimum of $f(x)$
\begin{equation}
    f^* = \max_{x\in\ccD} f(x) \ge \Exp{f(x)}{x\sim p(x|\theta)}
\end{equation}
where $p(x|\theta)$ is a probability distribution over $\ccD$ parameterized by $\theta$. 

\section{Differentiability of the variational bound}
We could apply REINFORCE technique to obtain an unbiased gradient estimator, i.e. 
\begin{align}
    \grad_{\theta}\Exp{f(x)}{x\sim p(x|\theta)} 
    &= \Exp{f(x)\grad_{\theta}\log p(x|\theta)}{x\sim p(x|\theta)} \\
    &\approx \avgsum{i}{1}{N}f(x_i)\grad_{\theta}\log p(x_i|\theta)
\end{align}

\section{Concavity of the variational bound}
\begin{definition}
    Let $I\subset\bbR$ be an interval in real line, a function $f: I\to\bbR$ is concave if $\forall x, y\in I$ and $\forall \alpha\in[0, 1]$ such that
    \begin{equation}
        f((1-\alpha)x + \alpha y) \ge (1-\alpha)f(x) + \alpha f(y). 
    \end{equation}
\end{definition}
Alternatively, one can rewrite it as
\begin{equation}
    f(x + \alpha(y - x)) \ge f(x) + \alpha(f(y) - f(x)). 
\end{equation}

\begin{definition}
    A probability distribution $p(x|\theta)$ parameterized by $\theta$ is expectation affine if
    \begin{equation}
        \Exp{f(x)}{x\sim p(x|\theta)} = \Exp{f(\alpha(\theta)z + \beta(\theta))}{z\sim q(z)}
    \end{equation}
    where $\alpha(\theta), \beta(\theta)$ are linear functions and $q(z)$ is a probability distribution. 
\end{definition}

\begin{theorem}\label{thm:concavebound}
    If a function $f(x)$ is concave and a probability distribution $p(x|\theta)$ parameterized by $\theta$ is expectation affine, then the variational lower bound $\Exp{f(x)}{x\sim p(x|\theta)}$ is concave in $\theta$. 
\end{theorem}
\begin{proof}
    Let $\lambda\in[0, 1]$ be a scalar. We have
    \begin{align*}
        \Exp{f(x)}{x\sim p(x|\lambda\theta_1 + (1-\lambda)\theta_2)}
        &= \Exp{f(\alpha(\lambda\theta_1 + (1-\lambda)\theta_2)z + \beta(\lambda\theta_1 + (1-\lambda)\theta_2))}{z\sim q(z)} \\
        &= \Exp{f(\lambda(\alpha(\theta_1)z + \beta(\theta_1)) + (1-\lambda)(\alpha(\theta_2)z + \beta(\theta_2)))}{z\sim q(z)} \\
        &\ge \Exp{\lambda f(\alpha(\theta_1)z + \beta(\theta_1)) + (1-\lambda) f(\alpha(\theta_2)z + \beta(\theta_2))}{z\sim q(z)} \\
        &= \lambda\Exp{f(\alpha(\theta_1)z + \beta(\theta_1))}{z\sim q(z)} + (1-\lambda)\Exp{f(\alpha(\theta_2)z + \beta(\theta_2))}{z\sim q(z)} \\
        &= \lambda\Exp{f(x)}{x\sim p(x|\theta_1)} + (1-\lambda)\Exp{f(x)}{x\sim p(x|\theta_2)}
    \end{align*}
\end{proof}
The \Thmref{thm:concavebound} indicates that if $f(x)$ has unique global optimum, so does its variational lower bound. 

\section{Estimation of distribution algorithm}
Given the optimization problem $f^* = \max_{x\in\ccD}f(x)$ where $f(x)$ is non-differentiable. We could instead optimize the variational lower bound $\Exp{f(x)}{x\sim p(x|\theta)}$. 

The general procedure is shown as following
\begin{enumerate}
    \item Prior distribution $p_0(x|\theta_0)$
    \item Iteratively
        \begin{enumerate}
            \item Generate solution set $\set{x_n}$ and evaluate them $\set{f(x_n)}$
            \item Update distribution: $p(x|\theta_{i+1})=F\paren{p(x|\theta_i), \set{x_n}, \set{f(x_n)}}$ for some function $F$. 
        \end{enumerate}
\end{enumerate}

\printbibliography

\end{document}